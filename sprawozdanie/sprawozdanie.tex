\documentclass[fleqn]{article}
\newcommand\x{30}

%\UseRawInputEncoding
%\usepackage[T1]{fontenc}
\usepackage[utf8]{inputenc}
\usepackage{polski}
%\usepackage[utf8]{inputenc}
%\usepackage{polski}
\usepackage[polish]{babel}

\usepackage{graphicx} 

\usepackage{float}

\usepackage{geometry}

\usepackage{mathtools}
\usepackage{amsmath}
\usepackage{physics}
\setlength\parindent{0pt}
\def\doubleunderline#1{\underline{\underline{#1}}}

\newcommand \A {1 - 0.8752 q^{-1}}
\newcommand \B {0.1248}
\newcommand \Am {1 - 0.693 q^{-1}}
\newcommand \Bm {3.684}
\newcommand \aOne {-0.8752}
\newcommand \amOne {-0.693}

\newcommand \bOne {0.1248} 
\newcommand \G {0.1594} 
\newcommand \F {1 + 0.1821 q^{-1}} 

\newcommand \R {\B + 0.0227 q^{-1}}
\newcommand \T {\Bm}
\begin{document}
  
\section{Założenia}

Obiekt sterowania:
\begin{equation}\label{eq:Gp}
G_p(s) = \frac{e^{-2s}}{1 + 15s}
\end{equation}

Wymagania:
\begin{equation}\label{eq:wymagania}
\begin{split}
& ks = 12 \\
& Tn = 15 \\
& Ts = 2
\end{split}
\end{equation}

Otrzymana transmitancja wzorcowa:
\begin{equation}\label{eq:tr_ref}
G_{ref}(s) = \frac{12}{5.455s + 1}
\end{equation}

\section{Dyskretyzacja}
Po dyskretyzacji metodą ZOH (zero-order hold) otrzymano transmitancje dyskretne:
\begin{equation}\label{eq:G_d}
\begin{split}
& G_p(q^{-1}) = q^{-2}\frac{\B}{\A} \\
& G_{ref}(q^{-1}) = q^{-1}\frac{\Bm}{\Am}
\end{split}
\end{equation} % G_p(q^{-1}) = q^{-1}\frac{0.1248q^{-1}}{1 - 0.8752q^{-1}}

Stąd:
\begin{equation}\label{eq:A_B_m}
\begin{split}
& A = \A \\
& B = \B \\
& A_m = \Am \\
& B_m = \Bm
\end{split}
\end{equation}

Więc zera i bieguny:
\begin{itemize}
	\item obiekt, transmitancja ciągła: zera: brak, biegun: $-0.0667$
	\item obiekt, transmitancja dyskretna: zero: $0+j0$, biegun: $0.8752$
	\item transmitancja wzorcowa, ciągła: zera: brak, biegun: $-0.1833$
	\item transmitancja wzorcowa dyskretna: zero: brak, biegun: $0.6930$
\end{itemize}

\begin{figure}[H]
    \centering
    \includegraphics[width=1\textwidth]{img/poles.png}
    \caption{Zera i bieguny układów}
    \label{fig:Poles}
\end{figure}


\section{Wielomiany sterownika pole placement i odpowiedź układu dyskretnego}

W celu uzyskania parametrów wzorcowego układu wykorzystano sterownik z pozycjonowaniem biegunów (ang. pole placement). Został on wybrany, gdyż umożliwia, w stosunkowo łatwy sposób, modyfikację układu, tak by uzyskać uzyskał on właściwości układu zadanego.   

Równanie diofantyczne: 
\begin{equation}\label{eq:dioph}
\begin{split}
& A_m(q^{-1}) A_0(q^{-1}) = A(q^{-1}) F(q^{-1}) + q^{-k} G(q^{-1}) \\
& (\Am) \times 1 = (\A) (1 + f_1 q^{-1}) + q^{-2} g_0 \\
\end{split}
\end{equation}

\begin{equation}\label{eq:dioph_sol}
\begin{split}
%& \amOne q^{-1} = \aOne q^{-1} + f_1 q^{-1} \\
%& 0 = \aOne f_1 q^{-2} + g_0 q^{-2} \\
& F(q^{-1}) = \F \\
& G(q^{-1}) = \G
\end{split}
\end{equation}

Stąd wielomiany $R$, $S$, $T$:
\begin{equation}\label{eq:polynomials}
\begin{split}
& S(q^{-1}) = G(q^{-1}) = \G \\
& R(q^{-1}) = B(q^{-1})F(q^{-1}) = \B (\F) = \R \\
& T(q^{-1}) = A_0(q^{-1})B_m(q^{-1}) = 1 \times \Bm = \Bm \\
\end{split}
\end{equation}

I transmitancje:
\begin{equation}\label{eq:transmitances}
\begin{split}
& \frac{T(q^{-1})}{R(q^{-1})} = \frac{\T}{\R} = \frac{3.684 z}{0.1248z + 0.0227}\\
& \frac{B(q^{-1})}{A(q^{-1})} q^{-k} = q^{-2}\frac{\B}{\A} = z^{-2} \frac{0.1248 z}{z - 0.8752}\\
& \frac{S(q^{-1})}{R(q^{-1})} = \frac{\G}{\R} = \frac{0.1594 z}{0.1248z + 0.0227}\\
\end{split}
\end{equation}

%\begin{equation}\label{eq:contr_signal}
%\begin{split}
%& R(q^{-1}) u(t) = T(q^{-1}) y_r(t) - S(q^{-1}) y(t) \\
%& (\R) u(t) = \Bm y_r(t) - \G y(t) \\
%& u(t) = ( \Bm y_r(t) - \G y(t) - 0.0277 u(t-1) ) / 0.1248
%\end{split}
%\end{equation}

%\begin{equation}\label{eq:output}
%\begin{split}
%& y(t+k) = \frac{B(q^{-1})}{A(q^{-1})} u(t) \\
%& y(t+k) A(q^{-1}) = B(q^{-1}) u(t) \\
%& y(t+k) (\A) = \B u(t) \\
%& y(t+2) = -(\aOne) y(t+1) + \B u(t) \\
%\end{split}
%\end{equation}
Wszystkie symulacje przeprowadzono przy pomocy programu Simulink.

Otrzymana odpowiedź jest bliska oczekiwanej. Wzmocnienie statyczne jest prawidłowe, czas narastania nieco dłuższy niż wymagany. Czas jest nieco dłuższy z racji na opóźnienie transportowe - na rys. \ref{fig:y_d_ref} widać, że sygnał uzyskany przy użyciu sterownika różni się od odpowiedzi skokowej zadanego układu jedynie opóźnieniem.
\begin{figure}[H]
    \centering
    %\includegraphics[width=1\textwidth]{img/u_d.png}
    \includegraphics[width=1\textwidth]{img/u_y_d.png}
    \caption{Sygnał wyjściowy i Sygnał sterujący - obiekt dyskretny}
    \label{fig:u_y_d}
\end{figure}

\begin{figure}[H]
    \centering
    %\includegraphics[width=1\textwidth]{img/u_d.png}
    \includegraphics[width=1\textwidth]{img/y_d_ref.png}
    \caption{Sygnał wyjściowy obiektu dyskretnego, razem z sygnałem pożądanego układu}
    \label{fig:y_d_ref}
\end{figure}

%\begin{figure}[H]
%    \centering
%    \includegraphics[width=1\textwidth]{img/y_d.png}
%    \caption{Wyjście i sygnał referencyjny - obiekt dyskretny}
%    \label{fig:y_d}
%\end{figure}




\section{Odpowiedź dla czasu ciągłego}

\subsection{Brak szumu pomiarowego}

Na rys. \ref{fig:y_c} widać, że zaprojektowany sterownik działa poprawnie również dla układu ciągłego. Na wykresie przedstawiono, dla odniesienia, przebieg sygnału dyskretnego z rys. \ref{fig:u_y_d}. Sygnał ciągły niewiele się od niego różni, a pomiędzy chwilami próbkowania nie ma niepożądanych oscylacji. 


\begin{figure}[H]
    \centering
    \includegraphics[width=1\textwidth]{img/y_c.png}
    \caption{Sygnał wyjściowy - obiekt ciągły (i dyskretny dla odniesienia)}
    \label{fig:y_c}
\end{figure}

\begin{figure}[H]
    \centering
    \includegraphics[width=1\textwidth]{img/u_c.png}
    \caption{Sygnał sterujący - obiekt ciągły}
    \label{fig:u_c}
\end{figure}


\subsection{Obecność szumu pomiarowego}
Odpowiedź układu na pobudzenie, przy obecności szumu pomiarowego o rozkładanie normalnym i wariancji $\delta_n=0.1$ przedstawiono na rys. \ref{fig:y_noise}. Szum nie wpływa znacząco na działanie obiektu.
\begin{figure}[H]
    \centering
    \includegraphics[width=1\textwidth]{img/y_noise.png}
    \caption{Sygnał wyjściowy - obiekt ciągły z zakłóceniami}
    \label{fig:y_noise}
\end{figure}


\begin{figure}[H]
    \centering
    \includegraphics[width=1\textwidth]{img/u_noise.png}
    \caption{Sygnał sterujący - obiekt ciągły z zakłóceniami}
    \label{fig:u_noise}
\end{figure}


\end{document}